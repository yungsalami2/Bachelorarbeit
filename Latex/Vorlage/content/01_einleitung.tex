\chapter{Einleitung}
Die Realisierung von Quantencomputer mit genügend hoher Anzahl an Qubits wird oft als das „heilige Gral der Wissenschaft“ angesehen, denn 
es soll nach theoretischen Überlegungen, die über ein halbes Jahrhundert zurückführen, in der Lage sein, die Grenzen moderner Computer zu 
überschreiten. \\
Im Gegensatz zum herkömmlichen Computer, der auf Basis von elektrischen Zuständen in Halbleitertransistoren funktioniert 
und nur die Zustände $\ket{0}$ oder $\ket{1}$ kennt, basiert der Quantencomputer auf der Superposition äquivalenter 
quantenmechanischer Zustände $\ket{0}$ und $\ket{1}$, wodurch u.a. eine drastisch höhere Anzahl von Zuständen auf gleichem Raum ermöglicht 
wird, was in bestimmten Fällen schnellere Prozesse zu Folge hat.\\
Um sich dieses Problem zu verdeutlichen, wird der Hilbertraum eines N-Teilchen-Systems bestehend aus ½-Spin-Teilchen betrachtet, welcher eine 
Dimension von $2^N$ besitzt. Allein die Speicherung aller Freiheitsgrade eines 100 Teilchen Systems ist eine unmögliche Herausforderung
derzeitig. \\
Dieses Problem hat schon Feynman zu der Annahme verleitet, dass wohl theoretisch ein Computer basierend 
auf quantenmechanischen Wirkungsmechanismen die wohl angebrachteste Lösung wäre. Und tatsächlich haben Bernstein and Vazirani 
den ersten Beweis veröffentlicht, dass ein Quantencomputer in der Lage ist, das exponentielle Wachstum der Rechendauer auf ein 
polynomiales Wachstum runterzubrechen\cite{10.1145/167088.167097}.\\
Darüber hinaus können viele Algorithmen implementiert werden, die vorher nicht möglich waren, dazu gehört auch der bekannte Shor-Algorithmus; 
ein Faktorisierungsverfahren, welches unter anderem ein gewaltiges Sicherheitsrisiko für das übliche RSA-Kryptosystem darstellt\cite{365700,10.1137/S0097539795293172}.\\

\noindent Die zentralen Bausteine eines Quantencomputers sind die Quantenbits, kurz Qubits. Dabei entspricht ein Qubit ein wohldefiniertes
Zwei-Niveau Quantensystem. In der Vergangenheit wurden bereits einige Realisierungsmöglichkeiten des Qubits umgesetzt wie Stickstoff-Fehlstellen-Zentren 
in Diamanten\cite{PhysRevLett.93.130501,Hanson2008-tn}, Kernspin von Molekülen in Flüssigkeiten\cite{RevModPhys.76.1037},
Supraleitende Schaltkreise\cite{RevModPhys.73.357} und Phosphorunreinheiten im Silikon\cite{Kane1998-pc}.\\
Der Realisierungsvorschlag, mit der sich diese Arbeit beschäftigt, ist den Spin eines in einem Quantenpunkt eingefangenen 
Elektrons zu verwenden\cite{Elzerman2004-wu,Bonadeo1998-nr,PMID:17901328,Fokina_2010,Spatzek2011-rn}. Um die Elektronenspindynamik zu beschreiben, wird 
das Zentralspinmodell verwendet und die Dynamik mit dem Variationsverfahren "Time-Dependent Variational Principle" (TDVP) behandelt.\\

