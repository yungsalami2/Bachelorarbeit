\chapter{Einleitung}
Die Realisierung von Quantencomputer mit genügend hoher Anzahl an Qubits wird oft als das „heilige Gral der Wissenschaft“ angesehen, denn 
es soll nach theoretischen Überlegungen, die über ein halbes Jahrhundert zurückführen, in der Lage sein, die Grenzen moderner Computer zu 
überschreiten. \\
Im Gegensatz zum herkömmlichen Computer, der auf Basis von elektrischen Zuständen in Halbleitertransistoren funktioniert 
und nur die Zustände $\ket{0}$ oder $\ket{1}$ kennt, basiert der Quantencomputer auf der Superposition äquivalenter 
quantenmechanischer Zustände $\ket{0}$ und $\ket{1}$, wodurch u.a. eine drastisch höhere Anzahl von Zuständen auf gleichem Raum ermöglicht 
wird, was in bestimmten Fällen schnellere Prozesse zu Folge hat.\\
Um sich dieses Problem zu verdeutlichen, wird der Hilbertraum eines N-Teilchen-Systems bestehend aus ½-Spin-Teilchen betrachtet, welches eine 
Dimension von $2^N$ besitzt; so dass allein die Speicherung aller Freiheitsgrade eines 100 Teilchen Systems eine unmögliche Herausforderung
derzeitig ist. \\
Dieses Problem hat schon Feynman zu der Annahme verleitet, dass wohl theoretisch ein Computer basierend 
auf quantenmechanischen Wirkungsmechanismen die wohl angebrachteste Lösung wäre. Und tatsächlich haben Bernstein and Vazirani 
den ersten Beweis veröffentlicht, dass ein Quantencomputer in der Lage ist, das exponentielle Wachstum der Rechendauer auf ein 
polynomiales Wachstum runterzubrechen.\\
Darüber hinaus können viele Algorithmen implementiert werden, die vorher nicht möglich waren, dazu gehört auch der bekannte Shor-Algorithmus; 
ein Faktorisierungsverfahren, welches unter anderem ein gewaltiges Sicherheitsrisiko für das übliche RSA-Kryptosystem darstellt.\\

\noindent Die zentralen Bausteine eines Quantencomputers sind die Quantenbits, kurz Qubits. Dabei entspricht ein Qubit ein wohldefiniertes
Zwei-Niveau Quantensystem. In der Vergangenheit wurden bereits einige Realisierungsmöglichkeiten des Qubits umgesetzt.\\
Der Realisierungsvorschlag, mit der sich diese Arbeit beschäftigt, ist den Spin eines in einem Quantenpunkt eingefangenen 
Elektrons zu verwenden. Um die Elektronenspindynamik zu beschreiben, wird das Central-Spin-Model verwendet und mit 
dem Variationsverfahren "Time-Dependent Variational Principle" (TDVP) behandelt.\\

\noindent Dabei wird durch Modifikationen einer zu variierenden klassischen bzw. semi-klassischen Wellenfunktion 
$\ket{\Psi} = \ket{\Psi(\mu_1,...,\mu_N)}$, das Problem in ein semi-klassisches Problem überführt. Dadurch soll die Problematik des exponentiell 
anwachsenden Hilbertraumes umgangen werden.
