\chapter{Einleitung}
Die Realisierung von Quantencomputer mit genügend hoher Anzahl an Qubits wird oft als das „heilige Gral der Wissenschaft“ angesehen, denn 
es soll nach theoretischen Überlegungen, die über ein halbes Jahrhundert zurückführen, in der Lage sein, die Grenzen moderner Computer zu 
überschreiten. \\Im Gegensatz zum herkömmlichen Computer, der auf Basis von elektrischen Zuständen in Halbleitertransistoren funktioniert 
und nur die Zustände $\ket{0}$ oder $\ket{1}$ kennt, basiert der Quantencomputer auf quantenmechanische Zustände, wodurch u.a. eine 
drastisch höhere Anzahl von Zuständen auf gleichem Raum ermöglicht wird, was in bestimmten schnellere Prozesse.\\
Um sich dieses Problem zu verdeutlichen, wird der Hilbertraum eines N-Teilchen-Systems bestehend aus ½-Teilchen betrachtet, welches eine 
Dimension von $2^N$ besitzt; so dass allein die Speicherung aller Freiheitsgrade eines 100 Teilchen Systems eine unmögliche Herausforderung
 derzeitig ist. Dieses Problem hat schon damalige Physiker zu der Annahme verleitet, dass wohl ein theoretisch ein Computer basierend 
 auf quantenmechanischen Wirkungsmechanismen die wohl angebrachteste Lösung wäre. Und tatsächlich haben Bernstein and Vazirani 
 den ersten Beweis veröffentlicht, dass ein Quantencomputer in der Lage ist, das exponentielle Wachstum der Rechendauer auf ein 
 polynomiales Wachstum runterzubrechen.\\
Darüber hinaus können viele Algorithmen implementiert werden, dazu gehört auch der bekannte Shor-Algorithmus; ein Faktorisierungsverfahren,
welches unter anderem ein gewaltiges Sicherheitsrisiko für das übliche RSA-Kryptosystem darstellt.\\

\noindent Die zentralen Bausteine eines Quantencomputers sind die Quantencubits, kurz Qubits. Damit werden beliebige quantemechanische 
zwei-Niveau-Systeme sammelbezeichnet. In der Vergangenheit wurden bereits einige Realisierungsmöglichkeiten des Qubits umgesetzt.\\
Die Realisierungsvorschlag mit der sich diese Arbeit beschäftigt, ist die des Spins eines in einem Quantenpunkt eingefangenen 
Elektrons (Loch). Um diesen Spin zu beschreiben wird das Central-Spin-Model zur Beschreibung der Elektronepindynamik verwendet und mit 
dem semi-klassischen Variationsverfahren "Time-Dependent Variational Principle" (TDVP) behandelt.\\

\noindent Die Hoffnung liegt darin, durch Modifikationen von Parametern $\mu$ einer zu variierenden Wellen 
$\ket{\Psi} = \ket{\Psi(\mu_1,...,\mu_N)}$ das Problem des exponentiell zu N anwachsenden Hilbertraumes zu umgehen und gegenüber 
herkömmlichen, vollkommen quantenmechanischem Ansätzen mit größeren N möglichst kleine Abweichungen von der exakten Lösung zu erhalten.
