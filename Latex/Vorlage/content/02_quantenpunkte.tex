\chapter{Quantenpunkte}

\noindent Diese Arbeit beschäftigt sich mit der mathematischen Modellierungen der Spindynamik eines Quantenpunktes, deshalb ist es 
sinnvoll vorab zu klären, was ein Quantenpunkt überhaupt ist. Damit einhergehend lassen sich die hier angenommenen Modelle begründen.\\



\begin{figure}[h!]
    \centering
    \includegraphics[width = 0.45\textwidth]{Abbildungen/QD_Schema.png}
    \includegraphics[width = 0.45\textwidth]{Abbildungen/einQD.png}
    \caption{Links:Schematischer Aufbau vom Wachstum eines InAs-Quantenpunkt auf einem GaAs-Substrat. Rechts: Automatisch 
    aufgelöstes Bild eines InAs-Quantenpunktes}
    \label{fig:QD_Schema}
\end{figure}

\noindent Die hier betrachten Quantenpunkte sind nanoskopische Halbleiterstrukturen, die wie ein dreidimensionaler Potentialtopf fungieren
 und einzelne Ladungsträger wie Elektronen oder Löcher einfangen können.\\
Die Stranski-Krastnov-Methode ist eine weitverbreitete Methode um Quantenpunkte zu erzeugen, dabei ist die Idee zwei Materialien mit 
unterschiedlicher Gitterkonstante mittels Molekularstrahlepitaxie aufeinander wachsen zu lassen. Der Kürze halber lässt es sich 
beispielhaft an den beiden Materialien InAs und GaAs erklären, denn dabei wird InAs-Schicht auf ein GaAs-Substrat wachsen gelassen, 
wobei die Gitterkonstante des InAs etwa 7\% größer ist, wodurch Spannung zwischen beiden Materialien entsteht. \\
\noindent Diese Spannung bewirkt Formationen von InAs-Inseln wie in \autoref{fig:QD_Schema} links zu sehen, die nur wenige zehn 
Nanometer groß sind, und verändert selbst die elektronische Eigenschaften, hauptsächlich den elektrische Feldgradienten des 
Quantenpunktes. Nun wird eine weiter Monoschicht GaAs auf das ganze Substrat gewachsen, so dass die InAs-Inseln vollständig von 
dem GaAs umschlossen sind. Diese umschlossenen InAs-Inseln bilden dann explizit den Quantenpunkt.\\


\begin{figure}
   \centering
    \includegraphics[width = 0.45\textwidth]{Abbildungen/QD_Energiebänder.png}
    \caption{Schematische räumliche Bandstruktur eines In(Ga)As/GaAs-Quantenpunktes mit den relevanten Bändern und einem 
    gefangenem Ladungsträger}
    \label{fig:QD_Bandstruktur}
\end{figure}
\noindent Das GaAs, welches den In(Ga)As Quantenpunkt umschließt, fungiert als Potentialtopf, denn die Energiebandlücke des InAs ist 
signifikant kleiner als die des GaAs. Nun kann ein einzelnes Elektron (Loch) räumlich einfangen werden und die Wechselwirkung mit dem 
Substrat unterdrückt werden, wodurch die Dekohärenzzeit deutlich verlängert werden kann. Eine kurze Dekohärenzzeit ist seit Beginn ein 
Hauptproblem bei der Realisierung des Quantencomputers. In \autoref{fig:QD_Bandstruktur} ist zu erkennen, dass einzelne Ladungsträger 
im InAs auf bestimmte Energieniveaus gefangen sind.\\
Dabei ist die Hyperfeinstruktur die dominante Wechselwirkung. Quantenpunkte werden deshalb auch als "künstliche Atome" bezeichnet,
 da ganz in Analogie zu Atomen ihre diskreten Enerigieniveaus von den eingefangen Teilchen besetzt werden können.
