\chapter{Zusammenfassung}
% Was wurde erreicht
Zusammenfassend konnte nachegewiesen werden, dass das TDVP mit dem klassischen Ansatz \ref{klassischer_Ansatz} die klassischen Lösungen reproduzieren 
konnte, in der die Spinlängen konstant bleiben. Dadurch ist zu erkennen, dass der vom klassischen Ansatz beschränkte Unterraum des Hilbertraumes zu klein
ist, um die exakte Dynamik wiedergeben zu können. Die Erweiterung auf $n$ 1/2-Spins folgt durch den Produktansatz immer in analogen Schritten.\\
Wird bei den Ergebnissen des modifizierten Ansatzes der Korrekturparameter $\mu_{12}=0$ gesetzt, zerfallen wie zu erwarten
die Gram-Matrix Gl. (\ref{Gram_Korrektur}) und partiellen Ableitungen des modifizierten Hamiltonians Gl. (\ref{dH_Q1}) bis Gl. (\ref{dH_Q12})
in die klassischen Lösungen.\\

Im Rahmen dieser Bachelorarbeit wurde die Lösung der Differentialgleichungen Gl. (\ref{QK_DGL1}) bis Gl. (\ref{QK_DGL3}) nicht behandelt. Es ist 
aber naheliegend, da der modifizierte Ansatz den ganzen Hilbertraum abdeckt, sich die exakte Lösung reproduzieren lässt. Dadurch würde sich 
mit Ausblick zukünftiger Arbeiten, die sich mit diesem Thema beschäftigen, die Überprüfung der Exaktheit dieser Lösung als zentrale Aufgabenstellung ergeben. \\
Für den Fall, dass die Exaktheit erfolgreich nachgewiesen wurde, wäre die Erweiterung auf 1/2-Spin-Systeme mit auf $n$ Spins die zentrale Aufgabe, 
denn dann dürften diese Korrekturparameter nicht mehr den gesamte Hilbertraum abdecken und Fehler sind höchstwahrscheinlich unvermeidbar. 
Da aber das zentrale Ziel den Rechenaufwand eines zu $n$ exponentiell anwachsenden Hilbertraumes zu umgehen ist, ist 
ein gewisser Fehler tolerierbar und eventuell durch eine besondere Wahl von Quantenkorrekturparameter minimierbar. Anknüpfend würde ein Vergleich mit 
anderen semiklassischen Ansätzen sinnvoll sein.
