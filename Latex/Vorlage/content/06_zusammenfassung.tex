\chapter{Zusammenfassung}
% Was wurde erreicht
Zusammenfassend konnte nachegewiesen werden, dass das TDVP mit dem klassischen Ansatz \ref{klassischer_Ansatz} die klassischen Lösungen reproduzieren 
konnte. Und es ist zu erkennen, dass eine Erweiterung auf N 1/2-Spins immer analoge Schritte zufolge hat.\\
Wird bei dem Ergebnissen des modifizierten Ansatzes der Korrekturparameter $\mu_{12}=0$ gesetzt, zerfallen wie zu erwarten
die Gram-Matrix \ref{Gram_Korrektur} und partiellen Ableitungen des modifizierten Hamiltonians \ref{dH_Q1} bis \ref{dH_Q12} in die klassische Lösungen.\\

Im Rahmen dieser Bachelorarbeit wurde die Lösung der Differentialgleichungen \ref{QK_DGL1} bis \ref{QK_DGL3} nicht behandelt. Es ist aber naheliegend, 
da der modifizierte Ansatz bis auf eine Hyperebene, die in einem entsprechendem Integral kein Gewicht findet, den gesamten Hilbertraum abdeckt, sich 
die exakte Lösung reproduzieren lässt. Dadurch würde sich mit Ausblick zukünftiger Arbeiten, die sich mit diesem Thema beschäftigen, die Überprüfung 
der Exaktheit dieser Lösung als zentrale Aufgabenstelllung ergeben. \\
Für den Fall, dass die Exaktheit erfolgreich nachgewiesen wurde, wäre die Erweiterung auf 1/2-Spin-Systeme mit mehr als zwei Spins die zentrale Aufgabe, 
denn dann dürften diese Korrekturparameter nicht mehr den gesamte Hilbertraum abdecken und Fehler sind höchstwahrscheinlich unvermeidbar. 
Da aber das zentrale Ziel den Rechenaufwand eines zu N exponentiell anwachsenden Hilbertraumes zu umgehen ist, ist 
ein gewisser Fehler tolerierbar und eventuell durch eine besondere Wahl von Quantenkorrekturparameter minimierbar. Anknüpfend würde ein Vergleich mit 
anderen semi-Klassischen Ansätzen sinnvoll sein.
