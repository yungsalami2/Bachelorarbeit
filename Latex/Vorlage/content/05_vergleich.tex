\chapter{Vergleich}
Hier wird nun die Klassische Lösung für den Zwei-Spin-Fall mit der exakten quantenmechanischen Lösung verglichen.
\section{quantenmechanische Lösung}
Für die exakte quantenmechanische Lösung, wird der Hamiltonian diagonalisiert über die Basis:
\begin{align}
    \ket{1} &= \ket{\uparrow\uparrow}   \\
    \ket{2} &= \ket{\downarrow\downarrow} \\
    \ket{3} &= \frac{1}{\sqrt{2}N_1}\left[(\epsilon_1+1)\ket{\uparrow\downarrow} +(\epsilon_1-1)\ket{\downarrow\uparrow} \right]\\
    \ket{4} &= \frac{1}{\sqrt{2}N_2}\left[(\epsilon_2+1)\ket{\uparrow\downarrow} +(\epsilon_2-1)\ket{\downarrow\uparrow} \right]
\end{align}
mit $\epsilon_{1,2} = \frac{\alpha \pm \sqrt{b^2(1-z)^2+\alpha^2} }{b(1-z)} $ und $N_{i} = \sqrt{\epsilon_i^2 + 1}$. Die dazugehörigen Eigenenergien lauten:
\begin{align}
    E_1 &= b(1+z) + \frac{\alpha}{4}\\
    E_2 &= -b(1+z) + \frac{\alpha}{4}\\
    E_3 &= -\frac{\alpha}{4} + \frac{\sqrt{\alpha^2 + b^2(1-z)^2}}{2}\\
    E_4 &= -\frac{\alpha}{4} - \frac{\sqrt{\alpha^2 + b^2(1-z)^2}}{2}\\
\end{align}
\noindent Über den Zeitentwicklungsoperator und einem beliebigen Startzustand $\ket{\Psi_0}$, lässt sich die Spindynamik hinschreiben als:
\begin{align}
    \bra{\Psi}\hat{S}_i\ket{\Psi} &= \sum_{i,j}e^{i(E_i-E_j)t}\bra{i}\hat{S}_i\ket{j}\bra{j}\ket{\Psi_0}\bra{\Psi_0}\ket{i}
\end{align}
\section{exemplarischer Vergleich}
Es wird der Startzustand $\ket{\Psi_0}=\ket{\uparrow\uparrow} + i\ket{\uparrow\downarrow} + \ket{\downarrow\uparrow} + 
i\ket{\downarrow\downarrow}$ verwendet, welches Äquivalent zu eine Elektronenspinerwartungswert entlang der X-Achse und orthogonal
dazu dem Kernspinerwartungswert entlang der Y-Achse, beide auf der X-Y-Ebene liegend. Unter all den möglichen Startzuständen gewählt,
da ....


\begin{figure}[h!]
    \centering
    \includegraphics[width = 0.45\textwidth]{Abbildungen/Plot_Sx.png}
    \includegraphics[width = 0.45\textwidth]{Abbildungen/Plot_Sy.png}
    \includegraphics[width = 0.45\textwidth]{Abbildungen/Plot_Sz.png}
    \caption{Spin-Erwartungswerte des Elektronenspins S (Rot) und des Kernspins (Grün) und der jeweils exakten quantenmechanischen Lösung
    (Schwarz) gegen die Zeit t aufgetragen für die Parameter: ...}
    \label{fig:Plots}
\end{figure}
